\documentclass[a4paper,12pt]{article}
\usepackage{xeCJK}
\input{/Users/yenchin/Documents/Seminar/P}
\setCJKmainfont{Songti SC}
\renewcommand{\baselinestretch}{1.5}
\usepackage{authblk}
\usepackage{geometry}
\geometry{a4paper, textheight=20.5cm, top=2.5cm, bottom=2cm, footskip=1.5cm, left=2.0cm, right=2.0cm}
\usepackage{fancyhdr}
\pagestyle{fancy}

\begin{document}
\title{Group Project Report\\IoT-based Virtual Health Determination}
\author{\footnotesize{蔡佳軒\;陳彥瑾\;黃柏淞\;陳威儒\;廖柏竣}}
\maketitle
\thispagestyle{empty}
\vspace{1cm}
\newpage

\setcounter{page}{1}

\section*{Shiny 使用手冊與操作說明}

本專案設計一個互動式的 Shiny 使用介面,搭配機器學習與深度學習模型,提供使用者簡易的震動訊號健康預測平台。以下說明使用方式與注意事項。

\subsection*{1. 檔案格式與上傳}

上傳之檔案需為純文字檔(.txt),內容為三欄數值資料,分別代表 X、Y、Z 軸的震動訊號,可選擇含有標題列或空一列略過。

\vspace{0.5em}
範例檔案內容如下:
\begin{verbatim}
0.0023  -0.0011  0.0005
0.0031  -0.0009  0.0004
0.0018  -0.0012  0.0006
\end{verbatim}

\subsection*{2. Shiny 預測流程}

\begin{itemize}
  \item 開啟 Shiny 頁面後,點選「學習方法」分頁。
  \item 選擇預測模型:CNN 或 Random Forest。
  \item 上傳測試檔案(例如:\texttt{test\_Xa\_sample.txt})。
  \item 選擇資料來源位置(Xa、Xb、Ya、Yb)。
  \item 若使用 CNN 模型,請額外選擇任務類型(連續型或離散型預測)。
  \item 點選「開始預測」後,系統將自動顯示下列結果:
  \begin{itemize}
    \item 資料歸屬位置與所使用模型類型
    \item 預測健康結果(健康 / 不健康 / 注意),並依機率加以顏色標示
    \item 預測負載數值及所對應之分類(是否屬正常負荷)
  \end{itemize}
\end{itemize}

\subsection*{3. 系統回饋與錯誤處理}

\begin{itemize}
  \item 若上傳之檔案資料列過短(如低於 5000 列),系統將自動濾除並提醒。
  \item 若欄位數非 3(如僅有 1 欄或超過 3 欄),系統將提示格式錯誤。
  \item 若檔案與選擇之資料來源(如 Xa 資料選 Yb 類別)不符,可能造成預測偏誤。
\end{itemize}

\subsection*{4. GitHub 使用說明}

若欲自行執行本系統,請至 GitHub 下載:

\begin{itemize}
  \item 專案主程式位於 \texttt{shiny\_bda} 專案資料夾中。
  \item 請務必同時下載相依的模型檔(如:\texttt{rf\_clf.pkl}, \texttt{*.pth}, \texttt{*.pkl})並放置於對應資料夾內,否則預測功能將無法運作。
  \item 使用方式為於 RStudio 中執行 \texttt{runApp()}。
\end{itemize}

\subsection*{5. 使用介面與貢獻摘要}

為提升使用者體驗,本專案特別設計一套清新簡潔、友善操作的使用者介面,整合前端 Shiny 與後端 Python 模型串接,讓非技術使用者也能輕鬆上手。

\textbf{專案貢獻簡述}:

\begin{quote}
本系統整合 Python 與 R 語言,開發了一個即時互動的預測介面。透過 Shiny 可視化平台,使用者只需上傳單筆震動資料,即可快速獲得設備健康狀態與負載預測。前端使用 R 語言設計使用者介面,後端由 Python 模型(隨機森林與 CNN)進行分析運算,並透過 JSON 溝通整合。系統支援自訂模型選擇、任務切換與即時錯誤提示,具備高度實用性與擴充彈性。
\end{quote}

\end{document}
